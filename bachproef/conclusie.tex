%%=============================================================================
%% Conclusie
%%=============================================================================

\chapter{Conclusie}
\label{ch:conclusie}

%% TODO: Trek een duidelijke conclusie, in de vorm van een antwoord op de
%% onderzoeksvra(a)g(en). Wat was jouw bijdrage aan het onderzoeksdomein en
%% hoe biedt dit meerwaarde aan het vakgebied/doelgroep? Reflecteer kritisch
%% over het resultaat. Had je deze uitkomst verwacht? Zijn er zaken die nog
%% niet duidelijk zijn? Heeft het onderzoek geleid tot nieuwe vragen die
%% uitnodigen tot verder onderzoek?

Om te kunnen concluderen wat nu de beste of veiligste beveiligingsmethodes is om een host-based card emulation applicatie te beveiligen moeten we alles in zijn geheel gaan bekijken. Als u voor het eerst nadenkt over hoe u moet bepalen wat nu de beste methode is om uw  applicatie te beveiligen denkt u waarschijnlijk direct aan hoe sterk de methode uw gegevens beveiligd. Maar om tot de beste oplossing te komen moeten we niet enkel naar de graad van veiligheid kijken maar ook naar de kost van implementatie en het opzetten van de beveiligingsmethode. 

Dit is ook een belangrijk gegeven om rekening mee te houden want niet iedere ontwikkelaar beschikt over evenveel resources maar niet iedere applicatie heeft een even ingewikkelde of dure implementatie nodig. Elke beveiligingsmethode heeft ook zijn voor en nadelen. Biometrische factoren hebben het voordeel dat de niet gemakkelijk na te bootsen zijn en dat de toegang tot de applicatie beveiligd is, het nadeel  is echter dat wanneer er gevoelige data verstuurd moet worden deze niet beveiligd is dus wanneer de data onderschept wordt de gegevens zomaar leesbaar zijn. Bij encyptie is het voordeel dat de gevoelige gegevens geëncrypteerd verstuurd worden en dus niet zomaar leesbaar zijn, de geëncrypeerde data kan enkel ontcijferd worden via de sleutel waarmee de data geëncrypteerd is. Het nadeel van encryptie is dat de toegang tot de applicatie niet beveiligd is. Het grote voordeel bij tokenisatie is dat de gevoelige data vervangen wordt door een onleesbare en ontcijferbaar token maar ook hier is de toegang tot de applicatie niet beveiligd. Nog een nadeel bij tokenisatie is de nood aan een server om de tokens te genereren en te bewaren wat een grote kost kan zijn voor sommige personen of bedrijven. 

Gelukkig kunnen de verschillende beveiligingsmehtodes gecombineerd worden voor extra veiligheid, zo kan de toegang tot de applicatie en de gevoelige data beveiligt worden. Hieruit kunnen we dus opmaken dat er niet één juiste of beste oplossing is voor alle gevallen. Voor betalingsapplicaties waar de veiligheid van data van groot belang is zal er best gekeken worden voor een implementatie met encryptie of tokenisatie of een combinatie van beiden. Wanneer het gaat over een kleine loyalty applicatie kan het gebruik van biometrische factoren of encryptie al voldoende zijn. Wanneer u dus beslist om uw HCE applicatie te beveiligen, houd dan zeker rekening met het doel van de applicatie en de kost van implementatie van de beveiligingsmethode.
