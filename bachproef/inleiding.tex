%%=============================================================================
%% Inleiding
%%=============================================================================

\chapter{Inleiding}
\label{ch:inleiding}

%%De inleiding moet de lezer net genoeg informatie verschaffen om het onderwerp te begrijpen en in te zien waarom de onderzoeksvraag de moeite waard is om te onderzoeken. In de inleiding ga je literatuurverwijzingen beperken, zodat de tekst vlot leesbaar blijft. Je kan de inleiding verder onderverdelen in secties als dit de tekst verduidelijkt. Zaken die aan bod kunnen komen in de inleiding~\autocite{Pollefliet2011}:

%%\begin{itemize}
%%  \item context, achtergrond
%%  \item afbakenen van het onderwerp
 %% \item verantwoording van het onderwerp, methodologie
 %% \item probleemstelling
 %% \item onderzoeksdoelstelling
%%  \item onderzoeksvraag
 %% \item \ldots
%%\end{itemize}

\section{Probleemstelling}
\label{sec:probleemstelling}

%%Uit je probleemstelling moet duidelijk zijn dat je onderzoek een meerwaarde heeft voor een concrete doelgroep. De doelgroep moet goed gedefinieerd en afgelijnd zijn. Doelgroepen als ``bedrijven,'' ``KMO's,'' systeembeheerders, enz.~zijn nog te vaag. Als je een lijstje kan maken van de personen/organisaties die een meerwaarde zullen vinden in deze bachelorproef (dit is eigenlijk je steekproefkader), dan is dat een indicatie dat de doelgroep goed gedefinieerd is. Dit kan een enkel bedrijf zijn of zelfs één persoon (je co-promotor/opdrachtgever).

Sinds de komst van Android4.4 (Kit Kat) is het mogelijk om smartcards te simuleren via een nieuwe technologie Host-based card emulation (HCE). HCE zorgt ervoor dat men smartcards die gebruikt worden voor betaalkaarten of loyalty kaarten kunnen simuleren zonder het gebruik van een Secure Element die aanwezig moet zijn in het mobiele toestel. Vroeger kon dit enkel aan de hand van het gebruik van een Secure Element, dit zorgde ervoor dat applicaties en gevoelige data veilig opgeslagen konden worden op het mobiele toestel. Het Secure Element is zowel hardware- als softwarematig beveiligd wat het zeer veilig maakt om gevoelige data in op te slaan, het nadeel en dus ook de nood voor HCE is de moeilijke samenwerking met de producenten van de Secure Elements. Een ontwikkelaar kan niet zomaar gebruikmaken van een Secure Element om zijn gegevens of applicatie in op te slaan, hiervoor moet er toestemming gevraagd worden aan de producent. Wanneer er toestemming gevraagd wordt zal er een commerciële overeenkomst getekend moeten worden. Door het tekenen van zo een overeenkomst is de ontwikkelaar sterk afhankelijk van de producent en zorgt dit voor een sterk verlaagde vrijheid voor de ontwikkelaar. Er bestaan natuurlijk meerdere producenten van Secure Elements, hierdoor zal de ontwikkelaar met meerdere producent een overeenkomst moeten sluiten om een oplossing te kunnen aanbieden voor zoveel mogelijk toestellen. Niet ieder toestel beschikt over een Secure Element wat de nood aan HCE ook verhoogt.

Door de komst van HCE is er geen nood meer aan een Secure Element en is men niet meer afhankelijk van de producent maar hierdoor zijn de applicatie en de gevoelige data niet goed meer beveiligd. Hierdoor kreeg ik de vraag van mijn stagebedrijf CCV Lab om een onderzoek te doen naar de verschillende mogelijkheden om een HCE applicatie te beveiligen. CCV Lab is een bedrijf die betalingsapplicaties, loyalty applicaties en kassasystemen ontwikkeld. Dit is dus een interessant onderzoek om een goede keuze te kunnen maken welke technologie ze kunnen gebruiken om hun applicaties te beveiligen. Dit geldt niet alleen voor het bedrijf CCV Lab maar ieder bedrijf of ontwikkelaar die bezig is met het ontwikkelen van een betalingsapplicatie aan de hand van HCE technologie. Door de steeds toenemende cyber aanvallen is er ook een grote nood aan beveiligingsmogelijkheden. Hierdoor zijn er tal van mogelijkheden beschikbaar om uw applicaties te beveiligen. Om te weten te komen welke beveiligingsmethode nu de beste is zullen er drie methodes uit gekozen worden waarvan een proof-of-concept uitgewerkt worden. Deze proof-of-concepts zullen dan afgetoetst worden op een aantal criteria om te bepalen wat nu de beste methodes is.

\section{Onderzoeksvraag}
\label{sec:onderzoeksvraag}

%%Wees zo concreet mogelijk bij het formuleren van je onderzoeksvraag. Een onderzoeksvraag is trouwens iets waar nog niemand op dit moment een antwoord heeft (voor zover je kan nagaan). Het opzoeken van bestaande informatie (bv. ``welke tools bestaan er voor deze toepassing?'') is dus geen onderzoeksvraag. Je kan de onderzoeksvraag verder specifiëren in deelvragen. Bv.~als je onderzoek gaat over performantiemetingen, dan 

In deze bachelorproef zal er onderzocht worden welke de verschillende mogelijkheden zijn om een android HCE applicatie te beveiligen en welkeen van deze mogelijkheden nu de beste is. Dit zal bepaald worden aan de hand van volgende criteria:

\begin{itemize}
	\item Moeilijkheidsgraad van de implementatie: Hoe ingewikkeld is het om een bepaalde methode te implementeren in de applicatie. Zijn er extra resources nodig voor de implementatie?
	\item Zichtbaarheid van de gevoelige gegevens: Zijn de gevoelige gegevens zomaar leesbaar, vanaf wanneer of tot wanneer zijn de gegevens zichtbaar?
	\item Veiligheid van de beveiligingsmethode: Hoe veilig zijn de methodes effectief, kunnen ze gemakkelijk omzeild worden?
	\item Combinatie van verschillende beveiligingsmethodes: Kunnen verschillende methodes gecombineerd worden met elkaar, zo ja welke combinaties kunnen er gemaakt worden? 
	\item Voordelen van het combineren van verschillende methodes: Wanneer verschillende methodes met elkaar gecombineerd worden heeft dit dan voordelen of ook nadelen? 
\end{itemize}



\section{Onderzoeksdoelstelling}
\label{sec:onderzoeksdoelstelling}

%%Wat is het beoogde resultaat van je bachelorproef? Wat zijn de criteria voor succes? Beschrijf die zo concreet mogelijk.

Op het einde van deze bachelorproef zou er een duidelijk beeld moet zijn van wat de verschillende methodes zijn om een applicatie met host-based card emulation te beveiligen. Daarnaast moet het ook duidelijk zijn wat de voor- en nadelen zijn van de verschillende methodes en wat de beste oplossing is in een specifieke use-case.

\section{Opzet van deze bachelorproef}
\label{sec:opzet-bachelorproef}

% Het is gebruikelijk aan het einde van de inleiding een overzicht te
% geven van de opbouw van de rest van de tekst. Deze sectie bevat al een aanzet
% die je kan aanvullen/aanpassen in functie van je eigen tekst.

De rest van deze bachelorproef is als volgt opgebouwd:

In Hoofdstuk~\ref{ch:stand-van-zaken} wordt een overzicht gegeven van de stand van zaken binnen het onderzoeksdomein, op basis van een literatuurstudie.

In Hoofdstuk~\ref{ch:methodologie} wordt de methodologie toegelicht en worden de gebruikte onderzoekstechnieken besproken om een antwoord te kunnen formuleren op de onderzoeksvragen.

% TODO: Vul hier aan voor je eigen hoofstukken, één of twee zinnen per hoofdstuk

In Hoofdstuk~\ref{ch:resultaten} worden de resultaten van het experiment besproken en antwoord gegeven op de onderzoeksvragen.

In Hoofdstuk~\ref{ch:conclusie}, tenslotte, wordt de conclusie gegeven en een antwoord geformuleerd op de onderzoeksvragen. Daarbij wordt ook een aanzet gegeven voor toekomstig onderzoek binnen dit domein.

